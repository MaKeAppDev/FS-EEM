\section{Synchronmaschine}
\begin{sectionbox}
\subsection{Größen}
\begin{tablebox}{p{4,5cm}cc}
Erregerstrom & $I_2$ & $\unitof{\si{\ampere}}$\\
induzierte Polradspannung & $\underline{U}_\text{iP}$ & $\unitof{\si{\volt}}$\\
synchrone Reaktanz & $X_d$ & $\unitof{\si{\ohm}}$\\
Selbstinduktivität & $L$ & $\unitof{\si{\henry}}$\\
Koppelinduktivität (von Rotor nach Stator) & $M_{21}$ & $\unitof{\si{\henry}}$\\
Polradwinkel & $\vartheta$ & $\unitof{\si{\radian}}$\\
Phasenwinkel von $\underline{Z}_1$ & $\varphi_{Z1}$ & $\unitof{\si{\radian}}$\\
\cmrule
Netzleistung (Wirkleistung) & $P_1$ & $\unitof{\si{\watt}}$\\
innere elektrische Leistung & $P_W$ & $\unitof{\si{\watt}}$\\
Drehfeldleistung & $P_\delta$ & $\unitof{\si{\watt}}$\\
mechanische Leistung & $P_m$ & $\unitof{\si{\watt}}$\\
Erregerleistung & $P_E$ & $\unitof{\si{\watt}}$\\
\cmrule
Leerlaufkurzschlussstrom & $\underline{I}_{K0}$ & $\unitof{\si{\ampere}}$\\
Dreisträngiger Dauerkurzschlussstrom & $\underline I_{K\textrm{III}}$ & $\unitof{\si{\ampere}}$\\
Leerlaufkurzschlussverhältnis (LKV) & $\frac{\underline{I}_{K0}}{\underline{I}_N}$ & $\unitof{\si{1}}$\\
\cmrule
Verketteter Fluss Permanentmagnet & $\underline{\Psi}_\text{PM}$ & $\unitof{\si{\voltsecond}}$\\
\end{tablebox}
\end{sectionbox}

\begin{sectionbox}
\subsection{ESB}
\begin{center}
\begin{circuitikz}[scale=.9, transform shape]
\draw [american currents]
	(0,2)
	to [short, i=$\underline{I}_1$, o-](1, 2) 
	to [R=\mbox{$R_1(= 0)$}](3,2)
	to [L=$jX_d$](5,2)
	to [I=$\underline{U}_\text{iP}$] (5,0)
	to [short, -o]		(0,0);
\draw[->, >=latex] (0,1.8) -- (0,0.2) node [pos=0.5, left] {$\underline U_1$};
\end{circuitikz}


\end{center}
\begin{align*}
\underline{U}_1 &= \underline{Z}_1\cdot\underline{I}_1 + \underline{U}_\text{iP}\\
\underline{Z}_1 &= R_1 + jX_d\\
X_d &= X_{1h} + X_{1\sigma} = 2\pi f\cdot (L_{1h} + L_{1\sigma})\\
\abs{\underline{U}_\text{iP}} &= U_\text{iP} = \omega M_{21}\sqrt{2}\cdot I_2\\
\sigma &= \frac{L_{1\sigma}}{L_{1h}}
\end{align*}

\subsection{Systemgleichungen}
\begin{align*}
\vec{u}_1 &= R_1\cdot\vec{i}_1(t) + \frac{\partial\vec{\Psi}_1(t)}{\partial t}\\
\vec{\Psi}_1 &= L_1\cdot\vec{i}_1(t) + M_{21}\cdot\vec{i}_2'(t)\\
u_2 &= R_2\cdot i_2(t) + \frac{\partial\Psi_2(t)}{\partial t}\\
\Psi_2 &= L_2\cdot i_2(t) + 3\cdot M_{21}\cdot(\vec{i}_1(t)e^{-jp\vartheta_m} + \vec{i}_1^*(t)e^{jp\vartheta_m})\\
\end{align*}
\end{sectionbox}

\begin{sectionbox}
\subsection{Wichtige Gleichungen}
\subsubsection{Synchrone Drehzahl Luftspaltfeld}
\begin{emphbox}
  $n_\text{syn} = n_N = \frac{f_1}{p}$
\end{emphbox}

\subsubsection{Drehmoment}
\begin{emphbox}
  $M_K\sim \frac{U_1}{f_1}$
\end{emphbox}
\begin{emphbox}
  \[M_i = -\frac{3p}{\omega_1}\cdot\left[\frac{U_1\cdot U_\text{iP}}{Z_1}\cdot\sin{\left(\vartheta - \varphi_{Z1}\right)} + \frac{{U_\text{iP}}^2}{Z_1}\cdot\sin{\left(\varphi_{Z1}\right)}\right]\]
\end{emphbox}
Kippmoment:
\[M_K = \frac{3p}{\omega_1}\cdot\frac{U_1\cdot U_\text{iP}}{Z_1} = \frac{3p}{\omega_1}\cdot U_1\cdot I_{K\textrm{III}}\]
$R_1 = 0 \Rightarrow \varphi_{Z1} = 0 \Rightarrow M_i = -M_K\cdot\sin(\vartheta)$

\subsubsection{Leistung}
\begin{align*}
\underline{S}_1 &= m_1\cdot \underline{U}_1\cdot\underline{I}_1^*\\
P_1 &= S_1 \cdot\cos{(\varphi)} = m_1\cdot U_1\cdot I_1\cdot\cos{(\varphi)}\\
P_W &= 3\cdot U_\text{iP}\cdot I_1\cdot\cos{(\varphi)}\\
P_\delta &= \omega_m\cdot M_i = P_W - 3\cdot R_1\cdot {I_1}^2\\
P_m &= 2\pi\cdot n\cdot (M_i - M_R) = \omega_m\cdot (M_i - M_R) = P_\delta - P_R\\
P_E &= U_2\cdot I_2\\
\eta &= \frac{P_m}{P_1 + P_{\text{v}E}}
\end{align*}
\end{sectionbox}

\begin{sectionbox}
\subsection{Betriebsbereiche}
Bei Leerlauferregung ($I_2 = I_{20}$): $\Rightarrow U_1 = U_\text{iP}$\\
Bei linearer Leerlaufkennlinie ($X_d = \const$): $I_2 = I_{20}\cdot\frac{U_\text{iP}}{U_1}$

\subsubsection{Leerlauf $(I_1 = 0)$}
\begin{align*}
I_{20} = \frac{U_\text{iP}}{\omega M_{21} \sqrt{2}} = \frac{U_1}{\omega M_{21} \sqrt{2}}
\end{align*}

\subsubsection{Kurzschluss $(U_1 = 0)$}
\begin{align*}
\underline I_{K\textrm{III}} &= \frac{\underline{U}_\text{iP}}{\underline{Z}_1}\\
\underline{I}_{K0} &= \underline I_{K\textrm{III}}(I_{20}) = \frac{\underline{U}_1}{\underline{Z}_1}
\end{align*}

\subsubsection{Betriebsarten}
\begin{symbolbox}
$\vartheta$ zwischen dem Zeiger von $\underline{U}_1$ nach $\underline{U}_\text{iP}$\\
$\varphi$ zwischen dem Zeiger von $\underline{I}_1$ nach $\underline{U}_1$\\
$\underline I_2$ eilt $\underline U_\text{iP}$ um $\ang{90}$ nach
\end{symbolbox}

\textbf{Phasenschieberbetrieb:} $\vartheta = 0$ ($R_1 = 0$ VZS - Betrieb am starren Netz)
\begin{itemize}
\item Betrieb im Leerlauf
\item reine Blindleistungsabgabe bzw. -aufnahme
\item $\cos(\varphi) = 0\Rightarrow$
\begin{itemize}
\item untererregt: $\Rightarrow\varphi = \SI{90}{\degree}$
\item übererregt: $\Rightarrow\varphi = \SI{-90}{\degree}$
\end{itemize}
\end{itemize}

\textbf{Motorbetrieb:} $\vartheta < 0$ ($R_1 = 0$ VZS - Betrieb am starren Netz)
\begin{center}
\begin{tikzpicture}[scale=.6, transform shape, font=\huge, >=angle 45]
	% UNTERERREGT
	\begin{scope}		
		\draw [very thick, red, ->] (0, 0) -- (65:1.5);
		\draw [very thick, ->] (0, 0) -- (0, 3);
		\draw [very thick, ->] (0, 0) -- (50:3);
		\draw [thick, ->, >=stealth] (0, 2.25) arc (90:50:2.25);
		\draw [very thick, blue, ->] (50:3) -- (0, 3);
		\draw [<-, >=stealth] (0, 1) arc (90:65:1);
		
		% BESCHRIFTUNGEN
		\draw (0,-0.5) node {untererregt $\varphi > 0$};
		
		\draw (70:2.5) node {$\vartheta$};
		\draw (-0.45, 2.6) node {$\underline U_1$};
		\draw (1.3, 3) node {$\underline U_\text{Xd}$};
		\draw (2.3, 2.6) node {$\underline U_\text{iP}$};
		\draw (65:1.75) node {$\underline I_1$};
		\draw (77:0.75) node{$\varphi$};
	\end{scope}
	
	% ÜBERERREGT
	\begin{scope}[shift={(5,0)}]
		\draw [very thick, red, ->] (0, 0) -- (120:1.5);
		\draw [very thick, ->] (0, 0) -- (0, 3);
		\draw [very thick, ->] (0, 0) -- (65:3.8);
		\draw [thick, ->, >=stealth] (0, 2) arc (90:65:2);
		\draw [very thick, blue, ->] (65:3.8) -- (0, 3);
		\draw [<-, >=stealth] (0, 1) arc (90:120:1);
		
		% BESCHRIFTUNGEN
		\draw (0,-0.5) node {übererregt $\varphi < 0$}; 
		
		\draw (77:1.75) node {$\vartheta$};
		\draw (-0.45, 2.6) node {$\underline U_1$};
		\draw (0.7, 3.55) node {$\underline U_\text{Xd}$};
		\draw (1.6, 2.5) node {$\underline U_\text{iP}$};
		\draw (110:1.75) node {$\underline I_1$};
		\draw (105:0.75) node{$\varphi$};
	\end{scope}
	
\end{tikzpicture}
\end{center}

\textbf{Generatorbetrieb:} $\vartheta > 0$ ($R_1 = 0$ VZS - Betrieb am starren Netz)
\begin{center}
\begin{tikzpicture}[scale=.6, transform shape, font=\huge, >=angle 45]
	
	% UNTERERREGT
	\begin{scope}
		\draw [very thick, ->] (0, 0) -- (0, 3);
		\draw [very thick, ->] (0, 0) -- (110:2.75);
		\draw [very thick, red, ->] (0, 0) -- (290:1.5);
		\draw [thick, ->, >=stealth] (0, 2) arc (90:110:2);
		\draw [very thick, blue, ->] (110:2.75) -- (0, 3);
		\draw [->, >=stealth] (290:1) arc (-70:90:1);
		
		%BESCHRIFTUNGEN
		\draw (0,-2) node {untererregt $\varphi > 0$}; 
		
		\draw (100:1.75) node {$\vartheta$};
		\draw (0.4, 2.6) node {$\underline U_1$};
		\draw (-0.85, 3.15) node {$\underline U_\text{Xd}$};
		\draw (-1.2, 2.0) node {$\underline U_\text{iP}$};
		\draw (0, -0.7) node {$\underline I_1$};
		\draw (0:0.75) node{$\varphi$};
	\end{scope}
	
	% ÜBERERREGT
	\begin{scope}[shift={(5,0)}]
		\draw [very thick, ->] (0, 0) -- (0, 3);
		\draw [very thick, ->] (0, 0) -- (110:3.75);
		\draw [very thick, red, ->] (0, 0) -- (225:1.5);
		\draw [thick, ->, >=stealth] (0, 2) arc (90:110:2);
		\draw [very thick, blue, ->] (110:3.75) -- (0, 3);
		\draw [->, >=stealth] (225:1) arc (225:90:1);
	
		%BESCHRIFTUNGEN
		\draw (0,-2) node {übererregt $\varphi < 0$};
		
		\draw (100:1.75) node {$\vartheta$};
		\draw (0.45, 2.7) node {$\underline U_1$};
		\draw (-0.5, 3.6) node {$\underline U_\text{Xd}$};
		\draw (-1.4, 3) node {$\underline U_\text{iP}$};
		\draw (0, -0.5) node {$\underline I_1$};
		\draw (150:0.75) node{$\varphi$};
	\end{scope}
	
\end{tikzpicture}
\end{center}
\end{sectionbox}

\begin{sectionbox}
\subsection{Zeigerdiagramm}
\begin{center}
%!tikz editor 1.0
%!tikz source begin
\begin{tikzpicture} [>=angle 45, font=\small]

% Koordinatensystem
\draw [->, thick] (2.5, 0) -- (-2, 0) node [left] {Im};
\draw [->, thick] (0, -0.5) -- (0, 4) node [above] {Re};

% Zeiger
\coordinate (U1) at (0,2);
\coordinate (Uip) at (2,3.5);
\coordinate (I1) at (-9/8,1.5);
\coordinate (I2) at (1,-0.57);
\coordinate (recht) at ($(0,0)!(Uip)!(I1)$);

\draw [black!50!white] (Uip) -- (recht);
\draw [->, very thick, green!70!blue] (Uip) -- (U1);
\draw [->, very thick, blue] (0, 0) -- (U1);
\draw [->, very thick, blue] (0, 0) -- (Uip);
\draw [->, very thick, red] (0, 0) -- (I1);
\draw [->, very thick, red] (0,0) -- (I2);

% Winkel
\draw [->, thick, >=stealth] (0, 0.8) arc (90:60:0.8);
\draw [<-, thick, >=stealth] (0, 0.8) arc (90:126:0.8);
\draw [black!50!white] (recht) ++(36:0.25) arc (36:-54:0.25);
\draw (recht) ++(-15:0.15) node [black!50!white] {.};
\draw [black!50!white] (0,0) ++(60:0.25) arc (60:-30:0.25);
\draw (0,0) ++(15:0.15) node [black!50!white] {.};

% Beschriftung
\draw (I1) ++(-0.1, -0.1) node [below] {$\underline I_1$};
\draw (U1) ++(0, -0.3) node [right] {$\underline U_1$};
\draw (U1) ++(1.6, 0) node {$\underline U_\text{iP}$};
\draw (Uip) ++(-1.3,-0.4) node {$jX_d\cdot\underline I_1$};
\node at (I2) [right] {$\underline I_2$};

\draw (75:0.8) node [above] {$\vartheta$};
\draw (110:0.8) node [above] {$\varphi$};
\end{tikzpicture}
%!tikz source end

\end{center}
\end{sectionbox}

\begin{sectionbox}
\subsection{Stromortskurve}
\begin{align*}
\underline{I}_1 &= \underline{I}_{K0} - \underline I_{K\textrm{III}}\\
\underline I_{K\textrm{III}} &= \frac{U_\text{iP}}{U_1}\cdot \underline{I}_{K0}\cdot e^{j\vartheta}\\
\underline{I}_{K0} &= -\frac{U_1}{Z_1}\cdot j\, e^{j\varphi_{Z1}}
\end{align*}
\begin{cookbox}{Stromortskurve}
\item $\underline{U}_1$ auf reelle Achse legen
\item Richtung von $\underline{U}_\text{iP}$ einzeichnen
\item $\underline{I}_{K0}$ einzeichnen\\
bei $R_1 = 0:\ \underline{I}_{K0}$ eilt $\underline{U}_1$ um $\ang{90}$ nach
\item konstante Erregung: Kreis um Spitze von $\underline{I}_{K0}$ mit Radius $I_{K\textrm{III}}$
\item Richtungen von $\underline{I}_{K\textrm{III}}$ und $\underline{I}_1$ festgelegt durch $\varphi$ bzw. $\vartheta$
\item bei $R_1 = 0$: Verlängerung von $\underline{U}_\text{iP} \perp \underline{I}_{K\textrm{III}}$
\end{cookbox}
\begin{center}
%!tikz editor 1.0
%!tikz source begin
\begin{tikzpicture} [scale=.6, >=angle 45, font=\small]

% Koordinatensystem
\draw [->, thick] (4, 0) -- (-4, 0) node [left] {Im};
\draw [->, thick] (0, -4) -- (0, 4) node [above] {Re};

% Zeiger
\draw [->, very thick, blue] (0, 0) -- (0, 2);
\draw [->, very thick, blue] (0, 0) -- (125:3.25);
\draw [->, very thick, blue] (125:3.25) -- (0, 2);

\draw (1.75, 0.75) circle (3.5);

\draw [->, very thick, red] (0, 0) -- (1.75, 0.75);
\draw [<-, very thick, red] (1.75, 0.75) -- ++(235:3.5);
\draw [<-, very thick, red] (1.75, 0.75) ++(235:3.5) -- (0, 0); 

\draw [->, >=stealth] (0, 1.65) arc (90:125:1.65);
\draw [->, >=stealth] (262:1.2) arc (262:90:1.2);
\draw [->, >=stealth] (1.75, 0.75) ++(235:1.25) arc (235:205:1.25);

\draw [dashed, thick, green!70!blue] (1.75, 0.75) ++(235:3.5) +(-3.8,0) node [left] {$P$} -- +(4.4, 0) ;

\draw [dashed, thick, red!50!yellow] (1.75, 3.8) -- (1.75, -2.7); 

% Beschriftungspfeile
\draw [<->, >=latex, thick] (2,0) -- (2, -2.1);
\draw [<->, >=latex, thick] (-2.75, 1) -- (-2.75, -1);
\draw [<->, >=latex, thick] (-1, -3.2) -- (1, -3.2);

% Beschriftung
\draw (1.6, 0.75) node [above] {\textbf M};
\draw (-1, -3.2) node [left, font=\large] {\begin{tabular}{c} übererregt \\ $I_{K\textrm{III}} > I_{K0}$ \end{tabular}};
\draw (1, -3.2) node [right, font=\large] {\begin{tabular}{c} untererregt\\ $I_{K\textrm{III}} < I_{K0}$ \end{tabular}};
\draw (-2.75, 1) node [above, font=\large] {Motor};
\draw (-2.75, -1) node [below, font=\large] {Generator};
\draw (2.1, -1) node [right, fill=gray!3] {$\text{Re}(\underline I_1) = |\underline I_1| \cdot \cos(\varphi)$};

\draw (0, 1.8) node [right] {$\underline U_1$};
\draw (-0.6, 2.2) node [above] {$\underline U_{Z1}$};
\draw (125:3.25) node [left] {$\underline U_\text{iP}$};

\draw (0.8, 0.8) node {$\underline I_{K0}$};
\draw (1.2, -1) node {$\underline I_{K\textrm{III}}$};
\draw (-0.4, -1.5) node {$\underline I_1$};

\draw (108:1.9) node {$\vartheta$};
\draw (145:1.4) node {$\varphi$};
\draw (0.5, -0.25) node {$-\vartheta$};

\draw (0.1, 2.5) node [right, blue, fill=gray!3] {\textbf{Spannungsdreieck}};
\draw (2, 0.5) node [right, red] {\textbf{Stromdreieck}};
\draw (1.7, 3.9) node {$P_W$ konstant};
\draw (1.75, 3.3) node [left, red!50!yellow] {stabil};
\draw (1.75, 3.3) node [right, red!50!yellow] {instabil}; 

\end{tikzpicture}
%!tikz source end

\end{center}
\end{sectionbox}

\begin{sectionbox}
\subsection{dq-Darstellung}
\begin{cookbox}{Zeigerdiagramm}
\item $\underline{U}_1$ auf reelle Achse legen
\item $\underline{I}_1$ einzeichnen
\item Richtung von $U_\text{iP}$ legt $d$ und $q$ Achse fest\\
($\vartheta =$ unbekannt $\Rightarrow$ weiter bei Trick)
\item Zerlegung von $\underline{I}_1$ in $\underline{I}_d$ und $\underline{I}_q$
\item Spannungsabfall an $X_d = \abs{X_d\cdot I_d}$
\item Spannungsabfall an $X_q = \abs{X_q\cdot I_q}$
\item $\underline{U}_\text{iP} = \underline{U}_1 - jX_d\cdot\underline{I}_d - jX_q\cdot\underline{I}_q$
\end{cookbox}
\begin{cookbox}{Trick}
\item $\vartheta = \arg(\underline{U}_1 - jX_q\cdot\underline{I}_1) \Rightarrow$ Richtungsgerade von $U_\text{iP} (||jX_d \underline I_d)$
\item $\underline{U}_\text{iP} =$ Senkrechte von $\underline{U}_1 - jX_d\cdot\underline{I}_d$ auf Richtungsgerade
\end{cookbox}

\subsubsection{Systemgleichungen}
\begin{align*}
U_d &= R_1\cdot I_d - \omega_1 L_q\cdot I_q\\
U_q &= R_1\cdot I_q + \omega_1 L_d\cdot I_d + \sqrt{2}\cdot U_\text{iP}\\
U_\text{iP} &= \sqrt{2}\cdot \omega_1 M_{21}\cdot I_2\\
U_2 &= R_2\cdot I_2\\
M_i &= 3\cdot p\cdot M_{21}\cdot I_2\cdot I_q
\end{align*}

\subsubsection{Zeigerdiagramm}
\begin{center}
%!tikz editor 1.0
%!tikz source begin
\begin{tikzpicture} [>=angle 45, font=\small]

% Koordinatensystem
\draw [->, thick] (3, 0) -- (-2, 0) node [left] {Im};
\draw [->, thick] (0, -1.5) -- (0, 5) node [above] {Re};

% Zeiger
\coordinate (U1) at (0,2);
\coordinate (Uip) at (60:4);
\coordinate (Iq) at (60:1.2);
\coordinate (I1) at (120:2.4);
\coordinate (q) at (240:1.5);
\coordinate (d) at (150:2);
\coordinate (recht) at ($(0,0)!(U1)!(I1)$);

\draw [->, very thick, dashed] (Uip) +(60:1) -- (q);
\draw [<-, very thick, dashed] (d) -- ++(-30:4);

\draw [black!50!white] (Uip) -- ++(-30:0.5);
\draw [black!50!white] (U1) -- +(210:1.8) ++(30:2.6) -- ++(30:0.3);

\draw [->, very thick, blue] (0, 0) -- (U1);
\draw [name path=Uline, ->, very thick, blue] (0, 0) -- (Uip);
\draw [->, very thick, red] (0, 0) -- (I1);
\draw [->, very thick, red] (0, 0) -- (Iq);
\draw [->, very thick, red] (Iq) -- (I1);

\draw [->, very thick, green!70!blue] (Uip) -- ++(150:1);
\draw [->, very thick, green!70!blue] (Uip) ++(150:1) -- (U1);
\draw [name path=Xline, <-, very thick, cyan] (U1) -- ++(30:2.6) node [midway, above, sloped, black] {$jX_q\underline I_1$};

\path [name intersections={of=Uline and Xline, by=A}];

% Winkel
\draw [->, thick, >=stealth] (0, 0.8) arc (90:60:0.8);
\draw [<-, thick, >=stealth] (0, 0.8) arc (90:120:0.8);
\draw [>=stealth, black!50!white] (recht) ++(30:0.25) arc (30:-60:0.25);
\draw (recht) ++(-15:0.15) node [black!50!white] {.};

% Beschriftung
\draw (d) node [left] {$d$};
\draw (q) node [left] {$q$};
\draw (Iq) ++(0, -0.2) node [right] {$\underline I_q$};
\draw (I1) ++(-0.1, -0.1) node [below] {$\underline I_1$};
\draw (I1) ++(0.25, -0.1) node [above] {$\underline I_d$};
\draw (U1) ++(0.1, -0.3) node [right] {$\underline U_1$};
\draw (U1) ++(1.6, 0) node {$\underline U_\text{iP}$};
\draw (Uip) ++(150:.8) node [above] {$jX_q \underline I_q$};
\draw (0.1, 3.5) node [fill=gray!3] {$jX_d\underline I_d$};
\draw (U1) ++(25:2) node [right] {$jX_d\underline I_1$};

\node at (A) [red, font=\large] {$\times$};
\draw [red] (A) ++(-0.1,0.2) node {A};
\draw (U1) ++(30:2.6) node [red, font=\large] {$\times$};
\draw (U1) ++(30:2.6) node [red, above] {B};

\node at (-2,4) [above] {$jX_d\underline I_1 = [BU_1]$};
\node at (-2,4) [below] {$jX_q\underline I_1 = [AU_1]$};

\draw (75:0.8) node [above] {$\vartheta$};
\draw (110:0.8) node [above] {$\varphi$};
\end{tikzpicture}
%!tikz source end

\end{center}
\end{sectionbox}

\begin{sectionbox}
\subsection{Schenkelpolläufer}
\subsubsection{Drehmoment ($R_1 = 0$)}
\begin{emphbox}
\[M_i' = -\frac{m_1\cdot p}{\omega_1} U_1\left[\frac{U_\text{iP}}{X_d}\sin(\vartheta) + \frac{U_1}{2}\left(\frac{1}{X_q} - \frac{1}{X_d}\right)\sin(2\vartheta)\right]\]
\end{emphbox}
\textbf{Reluktanzmoment (Reaktionsmoment):}
\begin{align*}
M_r = -\frac{m_1\cdot p}{\omega_1}\cdot\frac{{U_1}^2}{2}\left(\frac{1}{X_q} - \frac{1}{X_d}\right)\sin(2\vartheta)
\end{align*}
Vollpolläufer entwickeln kein Reluktanzmoment wegen $L_d = L_q$.\\
Maximales Reluktanzmoment bei $\abs{\vartheta} = \SI{45}{\degree}$.

\subsubsection{Systemgleichungen}
\begin{align*}
\underline{U}_1 &= \underline{U}_d + \underline{U}_q + \underline{U}_\text{iP}\\
&= jX_d\cdot\underline{I}_d + jX_q\cdot\underline{I}_q + \underline{U}_\text{iP}\\
\underline{I}_1 &= \underline{I}_d + \underline{I}_q
\end{align*}
\end{sectionbox}

\begin{sectionbox}
\subsection{Permanenterregte Synchronmaschine}
\subsubsection{Betriebsverhalten}
\begin{align*}
\underline{\Psi}_1 &= L_1\underline{I}_1 + \underline{\Psi}_\text{PM} & \Psi_\text{PM} &= \xi\cdot w_1 \cdot\hat{\Phi}_\delta\\
\Psi_d &= L_d I_d + \Psi_\text{PM} & \Psi_q &= L_q I_q\\
\end{align*}
\[U_\text{iP} = \sqrt{2}\omega_1\cdot\Psi_{PM}\]
\subsubsection{Drehmoment}
\begin{emphbox}
\[M_D = \frac{m_1}{2}\cdot p\cdot\left[\Psi_\text{PM}\cdot I_q + (L_d - L_q)\cdot I_d\,I_q\right]\]
\end{emphbox}
\textbf{äquvalent zu $M_D$ Schenkelpolläufer}

\subsubsection{Betriebsarten}
\begin{itemize}
\item symetrischer Betrieb
\item EC-Betrieb als BLDC
\end{itemize}
\end{sectionbox}
