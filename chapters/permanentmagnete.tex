\section{Permannentmagnete}
\sectionbox{
\subsection{Größen}
\tablebox{
\begin{tabular*}{\columnwidth}{p{4.2cm}cc}
\ctrule
Remanenzflussdichte & $B_r$ & $\unitof{\si{\tesla}}$\\
kritische Feldstärke (aus Kennlinie ablesen) & $H_\text{M,krit}$ & $\unitof{\si{\ampere\per\meter}}$\\
Steigung Scherungsgerade & $k_\text{SG}$ & $\unitof{\si{1}}$\\
Luftspalthöhe Permanentmagnet & $\delta_M$ & $\unitof{\si{\milli\meter}}$\\
Länge der Magnete & $l_M$ & $\unitof{\si{\meter}}$\\
Höhe Permanentmagnete & $h_M$ & $\unitof{\si{\meter}}$\\
\cbrule
\end{tabular*}
}
}
\sectionbox{
\subsection{Allgemein}
\begin{center}
%!tikz editor 1.0
%!tikz source begin
\begin{tikzpicture}[scale=.6,every node/.style={scale=.7}, >=stealth]

	\filldraw[fill=black!30!white, draw=black] (0,0) circle (3.6);
	\filldraw[fill=white, draw=black] (0,0) circle (3.1);
	
	\draw [thick] (0,0) circle (2.4);
	
	 \filldraw [fill=white!50!red] (90:3) arc (90:170:3) -- ++(0.5,0) 
		-- (167:2.5) arc (167:13:2.5) -- ($(10:3)+(-0.5,0)$) -- (10:3) arc (10:90:3);
	 
	 \filldraw [fill=white!50!green] (270:3) arc (270:190:3) -- ++(0.5,0) 
		-- (193:2.5) arc (193:347:2.5) -- ($(350:3)+(-0.5,0)$) -- (350:3) arc (350:270:3);
		
	
	\draw [draw=black!50!white, dashed, thick] (-3.7,0) -- (3.7,0) (0, 3.7) -- (0,-3.7);
	
	% beta Linien und Pfeile
	\draw (167:2.75) -- (0,0) -- (13:2.75);
	\draw [<->] (167:2.1) arc (167:13:2.1);
	
	\draw (163:2.75) -- (0,0) -- (17:2.75);
	\draw [<->] (163:1.2) arc (163:17:1.2);
	
	\draw (193:2.75) -- (0,0) -- (347:2.75);
	\draw [<->] (193:2.1) arc (193:347:2.1);

	\draw (197:2.75) -- (0,0) -- (343:2.75);
	\draw [<->] (197:1.2) arc (197:343:1.2);
	
	% Externe Linien und Pfeile
	% D_A1
	\draw (90:3.6) -- ++(6.2,0) (270:3.6) -- ++(6.2,0);
	\draw [<->] (6.1,3.6) -- (6.1,-3.6);
	
	% D_I1
	\draw (0,2.5) -- ++(5.2, 0) (0,-2.5) -- ++(5.2,0);
	\draw [<->] (5.1,2.5) -- (5.1,-2.5);
	
	% D_A2
	\draw (0,2.4) -- ++(4, 0) (0,-2.4) -- ++(4,0);
	\draw [<->] (3.9,2.4) -- (3.9,-2.4);
	
	% Rest
	\draw (0,3.1) -- ++(4, 0) (0,3) -- ++(4,0);
	\draw [<-] (2.4,3.6) -- ++(0,0.3);
	\draw [<-] (2.4,3.1) -- ++(0,-0.3);
	\draw [<->] (3.2,3) -- (3.2,2.5);
	\draw [<-] (3.2,3.1) -- ++(0,0.3);
	\draw [<-] (3.2,2.4) -- ++(0,-0.3);
	
	% Beschriftungen
	\draw (90:2.1) node [below] {$\beta_A$};
	\draw (90:1.2) node [below] {$\beta_M$};
	\draw (270:2.1) node [above] {$\beta_A$};
	\draw (270:1.2) node [above] {$\beta_M$};
	
	\draw (3.9,0) node [right] {$D_\text{A2}$};
	\draw (5.1,0) node [right=0.01cm] {$D_\text{I1}$};
	\draw (6.1,0) node [right=0.01cm] {$D_\text{A1}$};
	
	\draw (2.4,3.9) node [left] {$h_J$};
	\draw (3.2, 2.75) node [right] {$h_M$};
	\draw (3.2, 3.3) node [right] {$\delta_M$};
	\draw (3.2, 2.1) node [right] {$\delta$};
	
\end{tikzpicture}
%!tikz source end

\end{center}

\subsubsection{Flussdichte}
\begin{tabular}{ll}
Luftspalt & $B_\delta = -\mu_0 \frac{h_M}{\delta''}H_M = B_M \frac{A_M}{A_\delta}(1-\sigma)$\\
Permanentmagnet & $B_M = -\frac{h_M}{\delta''} \frac{A_\delta}{A_M} \frac{\mu_0}{1-\sigma} H_M = -k_\text{SG} \cdot H_M$
\end{tabular}

\subsubsection{Fluss}
\begin{tabularx}{\columnwidth}{lX}
Luftspalt & $\Phi_\delta = (1-\sigma)\Phi_M = B_\delta A_\delta$\\
Permanentmagnet & $\Phi_M = B_M A_M$
\end{tabularx}

\subsubsection{Fläche}
\begin{tabularx}{\columnwidth}{lX}
Luftspalt & $A_\delta = \beta_M \frac{D}{2}l_i = \beta_M\frac{D}{2}l_2\cdot k_\text{Fe}$\\
Permanentmagnet & $A_M = \beta_M \frac{D_\text{I1}}{2}l_M$\\
Leiterquerschnitt & $A_L = \frac{A_N \cdot k_Q}{Z_\text{N}}$
\end{tabularx}

\subsubsection{Materialgrößen}
\[\sigma = k_{\sigma 1}\cdot k_{\sigma 2}\]

\subsubsection{Effektiver Luftspalt}
\begin{tabularx}{\columnwidth}{>{\centering\arraybackslash}X >{\centering\arraybackslash}X}
$\delta' = k_{C2}\cdot(\delta + \delta_M)$ & \\
$\delta'' = (1 + k_\mu)$ & $k_\mu = \frac{V_\mu}{2\cdot H_\delta\cdot\delta'}$
\end{tabularx}
}
\sectionbox{
\subsection{Scherungsgerade}
\cookbox{Arbeitspunktbestimmung}{
\item Scherungsgerade: $B_M = -k_\text{SG} \cdot H_M$
\item Materialkennlinie: $B_M = \mu_0 \mu_r H_M + B_r$
\item Schneiden von Materialkennlinie und Scherungsgerade
\item $\Rightarrow$Arbeitspunkt: $H_M = -\frac{1}{\mu_0 \mu_r + k_\text{SG}} B_r$
}\\\\
Luftspaltfluss im Arbeitspunkt:\\
\[\Phi_{\delta P} = (1-\sigma)\cdot\frac{k_\text{SG}}{\mu_0 \mu_r + k_\text{SG}}\cdot B_r\cdot l_M\cdot\frac{D_\text{I1}}{2}\cdot\beta_M\]\\
Maximal zulässiger Ankerstrom:\\
\[I_\text{2,max} = \frac{\omega_2\cdot\beta_M}{2\pi\cdot (h_M+\delta'')} \cdot\Bigg\vert(H_M''-H_M) \frac{\mu_0\mu_r+k_\text{SG}}{k_\text{SG}}\frac{1}{1+\frac{\delta''}{h_M}}\Bigg\vert\]
Maximal zulässige Feldstärke: $H_M'' = \gamma_\text{krit} \cdot H_\text{M,krit}$
}
