\section{Wechselfeld - Drehfeld}
\sectionbox{
\subsection{Größen}
\tablebox{
\begin{tabular*}{\columnwidth}{p{4,5cm}cc}
\ctrule
Stator & Index 1 & \\
Rotor & Index 2 & \\
Ordnungszahl der Oberwellen & $\nu$ & $\unitof{\si{1}}$\\
\cmrule
elektrische Frequenz & $f$ & $\unitof{\si{\hertz}}$\\
elektrische Kreisfrequenz & $\omega$ & $\unitof{\si{\radian\per\second}}$\\
\multicolumn{3}{c}{$\omega = 2\pi f$}\\
mechanische Kreisfrequenz & $\omega_m$ & $\unitof{\si{\radian\per\second}}$\\
Phasenwinkel & $\varphi$ & $\unitof{\si{\radian}}$\\
Strangachsenwinkel & $\vartheta$ & $\unitof{\si{\radian}}$\\
Strangspannung & $U_1$ & $\unitof{\si{\volt}}$\\
Strangstrom & $I_1$ & $\unitof{\si{\ampere}}$\\
komplexe Scheinleistung & $\underline{S}$ & $\unitof{\si{\voltampere}}$\\
Wirkleistung & $P$ & $\unitof{\si{\watt}}$\\
Blindleistung & $Q$ & $\unitof{\si{\var}}$\\
Strangzahl & $m$ & $\unitof{\si{1}}$\\
Windungszahl pro Strang & $w_1$ & $\unitof{\si{1}}$\\
\cmrule
Lochzahl (Nuten pro Pol und Strang) & $q$ & $\unitof{\si{1}}$\\
Nutwinkel & $\alpha_N$ & $\unitof{\si{\radian}}$\\
Spulenwinkel & $\alpha_\text{Sp}$ & $\unitof{\si{\radian}}$\\
Polwinkel & $\alpha_p$ & $\unitof{\si{\radian}}$\\
Spulenweite & $W_\text{Sp}$ & $\unitof{\si{\centi\meter}}$\\
Zonungsfaktor & $\xi_Z$ & $\unitof{\si{1}}$\\
Sehnungsfaktor & $\xi_S$ & $\unitof{\si{1}}$\\
Nutschlitzbreitenfaktor & $\xi_N$ & $\unitof{\si{1}}$\\
Schrägungsfaktor & $\xi_\text{Schr}$ & $\unitof{\si{1}}$\\
\cbrule
\end{tabular*}
}
}
\sectionbox{
\subsection{Stern \& Dreieckschaltung}
\emphbox{
\begin{tabularx}{\columnwidth}{>{\centering\arraybackslash}X |>{\centering\arraybackslash}X}
\textbf{Sternschaltung} & \textbf{Dreiecksschaltung}\\
$U_1 = \frac{U_N}{\sqrt{3}}$ & $U_1 = U_N$\\
$I_1 = I_N$ & $I_1 = \frac{I_N}{\sqrt{3}}$\\
\end{tabularx}
}
}
\sectionbox{
\subsection{Allgemeines zu Wechselgrößen}
\symbolbox{
\begin{tabularx}{\columnwidth}{>{\centering\arraybackslash}X>{\centering\arraybackslash}X}
$\underline{a}^\nu = e^{j\nu\frac{2\pi}{3}}$ & $\underline{a}^0 + \underline{a}^1 + \underline{a}^2 = 0$\\
\multicolumn{2}{c}{$\underline{a}^2 = \underline{a}^* = e^{j\frac{4\pi}{3}} = e^{-j\frac{2\pi}{3}}$}
\end{tabularx}
}
\begin{align*}
x(t) &= \sqrt{2}\cdot X\cdot\cos(\omega t + \varphi)\\
\vec{x}(t) &= \frac{1}{3}\cdot\left[x_A(t) + \newvec{a}\cdot x_B(t) + \newvec{a}^2\cdot x_C(t) \right] = \frac{\sqrt{2}}{2}\cdot\underline X\cdot e^{j\omega t}\\
\underline{X} &= X\cdot e^{j\varphi}
\end{align*}

\subsubsection{Wechselfeld}
\begin{align*}
B(\vartheta,t) = \hat{B}\cdot\cos(\vartheta - \vartheta_0)\cdot\cos(\omega t - \varphi)
\end{align*}

\subsubsection{Drehfeld}
\begin{align*}
B(\vartheta,t) = \hat{B}\cdot\cos((\vartheta - \vartheta_0) - (\omega t - \varphi))
\end{align*}
}
\sectionbox{
\subsection{Einfluss realer Luftspalt}
\symbolbox{Wicklungsfaktor: $\qquad\xi_{(\nu)} = \xi_{Z(\nu)}\cdot\xi_{S(\nu)}\cdot\xi_{N(\nu)}$}
\[w_\text{eff} = w_\text{Sp}\cdot\xi_{(\nu)}\]
\begin{tabularx}{\columnwidth}{cCc}
$\alpha_N = \frac{2\pi}{N}$ & $\alpha_\text{Sp} = W_\text{Sp}(\text{absolut})\cdot\alpha_N$ & $\alpha_p = \frac{2\pi}{2p}$
\end{tabularx}

\subsubsection{Zonung}
Erhöhung der Lochzahl $q$\\
(Beschränkt durch $N_\text{max} = \frac{D\pi}{\tau_{N,\text{min}}}$) mit $\tau_{N,\text{min}} \approx \SI{1}{\centi\meter}$
\begin{align*}
w_\text{eff} &= q\cdot w_\text{Sp}\cdot\xi_{Z(\nu)}\\
\xi_{Z(\nu)} &= \frac{\sin\left(q\cdot\nu\frac{\alpha_N}{2}p\right)}{q\cdot\sin\left(\nu\frac{\alpha_N}{2}p\right)} = \frac{\sin\left(\nu\frac{\pi}{2}\frac{q}{Q}\right)}{q\cdot\sin\left(\nu\frac{\pi}{2}\frac{1}{Q}\right)}
\end{align*}

\subsubsection{Sehnung}
Kürzung der Spulenweite $W_\text{Sp}$ (nicht bei Einschichtwicklung möglich)\\
\begin{align*}
w_\text{eff} &= q\cdot w_\text{Sp}\cdot\xi_{S(\nu)}\\
\xi_{S(\nu)} &= \sin\left(\nu\frac{\pi}{2}\frac{W_\text{Sp}}{\tau_p}\right) = \sin\left(\nu\frac{\alpha_\text{Sp}}{\alpha_p}\frac{\pi}{2}\right)
\end{align*}

\subsubsection{Nutschlitzbreite}
\begin{align*}
w_\text{eff} &= w_\text{Sp}\cdot\xi_{N(\nu)}\\
\xi_{N(\nu)} &= \frac{\sin\left(\nu\frac{b_N}{D}\right)}{\nu\frac{b_N}{D}}
\end{align*}
}
