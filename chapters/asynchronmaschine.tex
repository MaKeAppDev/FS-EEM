\section{Asynchronmaschine}
\begin{sectionbox}
\subsection{Größen}
\begin{tablebox}{p{4,5cm}cc}
Übersetzungsverhältnis & $\ddot{u}$ & $\unitof{\si{1}}$\\
Schlupf & $s$ & $\unitof{\si{1}}$\\
Kippschlupf & $s_K$ & $\unitof{\si{1}}$\\
Kippmoment & $M_K$ & $\unitof{\si{\newtonmeter}}$\\
\cmrule
Bezogener Statorwiderstand & $\rho_1$ & $\unitof{\si{1}}$\\
Bezogener Rotorwiderstand & $\rho_2$ & $\unitof{\si{1}}$\\
Hilfsgröße & $\Delta\rho_1$ & $\unitof{\si{1}}$\\
\cmrule
Rotor-Statorwärmeverluste & $P_\text{Cu}$ & $\unitof{\si{\watt}}$\\
Magnetisierungsstrom & $\underline{I}_{1\mu}$ & $\unitof{\si{\ampere}}$\\
Rotor-Vorwiderstand & $R_{2V}$ & $\unitof{\si{\ohm}}$\\
\end{tablebox}
\end{sectionbox}

\begin{sectionbox}
\subsection{ESB}
\begin{center}
\begin{circuitikz}[scale=.75, transform shape, font=\large]
\draw [american currents]
	(0,2)
	to [short, i=$\underline{I}_1$, o-](.5, 2)
	to [R=\mbox{$R_\text{Fe}=\frac{3\cdot{U_1}^2}{P_\text{Fe}}$}, *-*] (.5,0)
	(.5,2)
	to [R=\mbox{$R_1(=0)$}](2,2) 
	to [L=$j\omega_1L_{1\sigma}$](4,2)
	to [L=$j\omega_1L_{1h}$, i=$\underline{I}_{1\mu}$, v>=$\underline{U}_{1i}$, *-*] (4,0)
	(4,2)
	to [L=$j\omega_1L_{2\sigma}'$](6,2)
	to [R=$R_{2,\text{ges}}'$, i<=$\underline{I}_2'$](8,2)
	to [R, l_=$R_{2,\text{ges}}'\cdot\frac{1-s}{s}$] (8,0) 
	to [short, -o]	(0,0);
\draw[->, >=latex] (0,1.8) -- (0,0.2) node [pos=0.5, left] {$\underline U_1$};
\end{circuitikz}


\end{center}

\subsubsection{Übersetzungsverhältnis}
Bei Schleifring-ASM gilt: $\quad M_{21} = M_{12} = M$

\begin{symbolbox}
\begin{center}
$\ddot{u} = \dfrac{L_{1h}}{M} = \sqrt{\frac{m_1}{m_2}}\cdot\frac{w_1\xi_1}{w_2\xi_2}\cdot\frac{1}{\xi_\text{Schr}} = \sqrt{\frac{m_1}{m_2}}\cdot\frac{w_{1,\text{eff}}}{w_{2,\text{eff}}}\cdot\frac{1}{\xi_\text{Schr}}$
\end{center}
\end{symbolbox}
\begin{tabularx}{\columnwidth}{CC}
$R_{2,\text{ges}}' = \ddot{u}^2\cdot R_{2,\text{ges}}$ & $R_{2,\text{ges}}' = R_2' + R_{2V}'$\\
$\underline U_2 = \frac{1}{\ddot{u}}\cdot\underline U_{1i}$ &  $L_{2\sigma}' = \ddot{u}^2\cdot (L_{2\sigma} + L_{2\text{Schr}})$\\
$\underline{I}_2' = \frac{1}{\ddot{u}}\cdot\underline{I}_2$ &
\end{tabularx}

\subsection{Systemgleichungen}
\begin{align*}
\vec{u}_1 &= R_1\cdot\vec{i}_1 + \frac{\partial\vec\Psi_1}{\partial t}, & \vec\Psi_1 &= L_1\cdot\vec i_1 + M\cdot\vec i_2\cdot e^{jp\vartheta_m}\\
0 &= R_{2,\text{ges}}\cdot\vec{i}_2 + \frac{\partial\vec\Psi_2}{\partial t}, & \vec\Psi_2 &= L_2\cdot\vec i_2 + M\cdot\vec i_1\cdot e^{-jp\vartheta_m}\\
J\frac{\diff\omega}{\diff t} &= M_i - M_R - M_L
\end{align*}
\end{sectionbox}

\begin{sectionbox}
\subsection{Wichtige Größen}
\subsubsection{Schlupf}
\begin{emphbox}
$s = \frac{n_\text{syn} - n}{n_\text{syn}} = \frac{\omega_\text{syn} - \omega_m}{\omega_\text{syn}} = \frac{\omega_1 - p\cdot\omega_m}{\omega_1} = \frac{\omega_2}{\omega_1}$
\end{emphbox}
\begin{tabularx}{\columnwidth}{CCC}
Gegenstrombremse & Motor & Generator\\
$s > 1$ & $1 > s > 0$ & $s < 0$
\end{tabularx}

\subsubsection{Drehzahl}
\begin{tabularx}{\columnwidth}{CC}
synchrone Drehzahl & Nenndrehzahl\\
$n_\text{syn} = \frac{f}{p}$ & $n_N = n_s (1-s_N)$
\end{tabularx}

\subsubsection{Leistung}
\begin{align*}
\underline{S}_1 &= m_1\cdot \underline{U}_1\cdot\underline{I}_1^*\\
P_1 &= S_1 \cdot\cos{(\varphi)} = m_1\cdot U_1\cdot I_1\cdot\cos{(\varphi)}\\
P_\text{Netz} &= m_1\cdot U_1\cdot I_1\cdot\cos{(\varphi_N)} = P_1 + P_\text{Fe}\\
P_\delta &= 2\pi\cdot n_\text{syn}\cdot M_i = P_1 - P_{\text{Cu}1} - P_\text{Fe}\\
P_{mi} &= (1-s)P_\delta = P_\delta - P_{\text{Cu}2} - P_{2V} = \omega_m\cdot M_i\\
P_m &= 2\pi\cdot n\cdot (M_i - M_R) = \omega_m\cdot (M_i - M_R) = P_{mi} - P_R\\
P_{\text{Cu}2} &= s\cdot P_\delta = m_2\cdot R_2\cdot {I_2}^2\\
\end{align*}

\subsubsection{Phase}
\begin{emphbox}
ASM immer induktiv $\Rightarrow \varphi > 0$
\end{emphbox}
\begin{align*}
\varphi &= \varphi_{1Z}-\varphi_{1N}\\
\varphi &= \begin{cases}
\arctan(\frac{b}{a}) & \text{für } a > 0\\
\arctan(\frac{b}{a})+\pi & \text{für } a < 0, b\geq 0\\
\arctan(\frac{b}{a})-\pi & \text{für } a < 0, b<0
\end{cases}
\end{align*}
\end{sectionbox}

\begin{sectionbox}
\subsubsection{Weitere Parameter}
\begin{tabularx}{\columnwidth}{CC}
$L_{1\sigma} = \sigma_1\cdot L_{1h}$ & $L_1 = L_{1h} + L_{1\sigma}$\\
$L_{2\sigma}' = \sigma_2\cdot L_{1h}$ & $L_2' = L_{1h}\cdot (1+\sigma_2)$\\
\multicolumn{2}{c}{$L_\sigma = \sigma\cdot L_1 = L_{1\sigma} + \frac{\xi_\text{Schr}}{1 + \sigma_2}L_{2\sigma}'$}\\
$\rho_1 = \frac{R_1}{\omega_1 L_1}$ & $\rho_2 = \frac{R_{2,\text{ges}}}{\omega_1 L_2} = \frac{R_{2,\text{ges}}'}{\omega_1 L_2'}$\\
\multicolumn{2}{c}{$\Delta \rho_1 = \sqrt{1+\left( \frac{\rho_1}{\sigma}\right)^2}\cdot\sqrt{1+{\rho_1}^2}$}\\
\multicolumn{2}{c}{$\sigma = 1 - \frac{1}{(1 + \sigma_1)\cdot(1 + \sigma_2)} = 1 - \frac{M^2}{L_1 L_2}$}
\end{tabularx}
\end{sectionbox}

\begin{sectionbox}
\subsection{Statorstrom}
\begin{emphbox}
\[\underline{I}_1 = \frac{\underline{U}_1}{\omega_1 L_1} \cdot \frac{\rho_2 +js}{\rho_1\cdot\rho_2 -\sigma\cdot s+j(\rho_2 +s\cdot \rho_1)}\]
\end{emphbox}
Anlaufstrom:\\
$I_{1A} = |\underline{I}_1|(s=1) = \frac{U_1}{\omega_1 L_\sigma}\sqrt{\frac{1+{\rho_2}^2}{\left(1-\frac{\rho_1\cdot\rho_2}{\sigma}\right)^2 +\left(\frac{\rho_1+\rho_2}{\sigma}\right)^2}}$\\
Ideeller Kurzschlussstrom:\\
$I_{1Ki} = |\underline{I}_1|(s\rightarrow\pm\infty) = \frac{U_1}{\omega_1 L_\sigma}\cdot \frac{1}{\sqrt{1+\left( \frac{\rho_1}{\sigma}\right)^2}}$\\
Leerlaufstrom:\\$I_{10} = |\underline{I}_1|(s=0) = \frac{U_1}{\omega_1 L_1}\cdot \frac{1}{\sqrt{1+{\rho_1}^2}}$

\subsubsection{Magnetisierungsstrom}
\[\underline I_\mu = \frac{\rho_2 + j\cdot s\cdot(\sigma - \sigma_1\cdot(1-\sigma))}{\rho_1\cdot\rho_2 - \sigma\cdot s + j\cdot(\rho_2 + s\cdot\rho_1)}\cdot\frac{\underline U_1}{\omega_1 L_1}\]
\end{sectionbox}

\begin{sectionbox}
\subsection{Zeigerdiagramm}
\begin{cookbox}{Zeigerdiagramm}
\item $\underline{U}_1$ auf reelle Achse legen und $\underline{I}_1$ einzeichnen
\item $R_1\underline{I}_1$ (gleiche Phasenlage wie $\underline{I}_1$)\\
$j\omega_1 L_{1\sigma}\underline{I}_1$ (eilt $\underline{I}_1$ um $\ang{90}$ voraus)
\item $\underline{U}_{1i} = \underline{U}_1 - R_1\underline{I}_1 - j\omega_1 L_{1\sigma}\underline{I}_1$
\item $\underline{I}_{1\mu} = \frac{\underline{U}_{1i}}{j\omega_1 L_{1h}}$ (eilt $\underline{U}_{1i}$ um $\ang{90}$ nach)
\item $\underline{I}_2' = \underline{I}_{1\mu} - \underline{I}_1$
\item $R_{2,\text{ges}}'\underline{I}_2'$ (parallel zu $\underline{I}_2'$)
\item $j\omega_1 L_{2\sigma}'\underline{I}_2'$ (eilt $\underline{I}_2'$ um $\ang{90}$ voraus)
\item $R_{2,\text{ges}}'\cdot\frac{1-s}{s}\cdot\underline{I}_2' = -\underline{U}_{1i} - R_{2,\text{ges}}'\,\underline{I}_2' - j\omega_1 L_{2\sigma}'\underline{I}_2'$
\end{cookbox}
\begin{center}
%!tikz editor 1.0
%!tikz source begin
\begin{tikzpicture} [>=angle 45, font=\small]

% Koordinatensystem
\draw [->, thick] (-0.5, 0) -- (2, 0) node [right] {$-$Im};
\draw [->, thick] (0, -0.5) -- (0, 4) node [above] {Re};

% Koordinaten
\coordinate (U1) at (0,3.7);
\coordinate (I1) at (57:2.5);
\coordinate (Ui1) at (82:2.9);
\coordinate (I1u) at (-8:0.7);

\coordinate (P1) at ($(0,0)!(U1)!(I1)$);
\coordinate (P2) at ($(I1u)!(Ui1)!(I1)$);
\coordinate (P3) at ($(P2)!(0,0)!(I1)$);
\coordinate (P4) at ($(U1)!(Ui1)!(P1)$);
\coordinate (P5) at ($(Ui1)!(0,0)!(P2)$);

\draw [black!50!white] (0,0) -- (P1);
\draw [black!50!white] (P1) -- (U1);
\draw [black!50!white] (Ui1) -- (P2);
\draw [black!50!white] (I1) -- (P2);

\draw [->, very thick, blue] (0, 0) -- (U1);
\draw [->, very thick, red] (0, 0) -- (I1);
\draw [->, very thick, blue] (0, 0) -- (Ui1);
\draw [->, very thick, red] (0, 0) -- (I1u);
\draw [->, very thick, red] (I1) -- (I1u);

\draw [->, thick, green!70!blue] (Ui1) -- (P4);
\draw [->, thick, green!70!blue] (P4) -- (U1);
\draw [->, thick, green!70!blue] (Ui1) -- (P5);
\draw [->, very thick, green] (P5) -- (0, 0);
\draw [->, very thick, green!70!blue] (P5) -- ($(P5)!0.2!(0,0)$);

% Beschriftungspfeile
\draw [->, >=stealth] (-0.25, 2.4) -- ($(0,0)!0.8!(Ui1)$);
\draw [->, >=stealth] (-0.25, 3.2) -- ($(P4)!0.5!(Ui1)$);
\draw [->, >=stealth] (1.6, 3.2) -- ($(Ui1)!0.2!(P5)$);
\draw [->, >=stealth] (1.9, 2.75) -- ($(P5)!0.1!(0,0)$);
\draw [->, >=stealth] (1.9, 1.45) -- ($(P5)!0.5!(0,0)$);

% Beschriftung
\node [left] at (0, 1.5) {$\underline U_1$};
\node [right=2] at (I1) {$\underline I_1$};
\draw (I1u) +(-0.2,0) node [below] {$\underline I_{1\mu}$};
\draw (I1u) +(0.2,0.5) node [right] {$\underline I_{2}'$};
\node [left] at (-0.25, 2.4) {$\underline U_{1i}$};
\draw ($(P4)!0.5!(U1)$) +(0,0.2) node [right] {$j\omega_1 L_{1\sigma} \cdot\underline I_1$};
\draw (-0.25, 3.2) node [left] {$R_1\cdot \underline I_1$};
\draw (1.6, 3.2) node [right] {$j\omega_1 L_{2\sigma}' \cdot \underline I_2'$};
\draw (1.9, 2.75) node [right] {$R_{2\text{ges}}'\cdot\underline I_2'$};
\draw (1.9, 1.45) node [right] {$R_{2\text{ges}}'\cdot \frac{1-s}{s}\cdot\underline I_2'$};
\end{tikzpicture}
%!tikz source end

\end{center}
\end{sectionbox}

\begin{sectionbox}
\subsection{Stromortskurve}
\begin{symbolbox}
  bei $R_1 = 0\qquad\qquad\qquad\tan(\mu) = s_K$
\end{symbolbox}
\begin{cookbox}{Stromortskurve $R_1 = 0\wedge R_\text{Fe} = 0$}
\item $\underline{U}_1$ auf reelle Achse legen $\Rightarrow\varphi_{1U} = 0$
\item $R_1 = 0\Rightarrow \underline I_{10}$ und $\underline I_{1Ki}$ haben keinen Realteil
\item Kreismittelpunkt auf Im-Achse zwischen $\underline{I}_{1Ki}$ und $\underline{I}_{10}$
\item $\mu$ zwischen $P_0$ und $P_A$
\end{cookbox}
%!tikz editor 1.0
%!tikz source begin
\begin{tikzpicture}[>=angle 60, font=\small]

% Koordinatensystem
\draw [->, thick] (-.4, 0) -- (6, 0) node [above] {$-$Im};
\draw [->, thick] (0, -2) -- (0, 2.5) node [right] {Re};

% Kreis
\coordinate (M) at (3, 0);
\draw [thick] (M) circle (2);

% Vektoren
\coordinate (U1) at (0, 1.75);
\coordinate (Iki1) at ($(M)+(2,0)$);
\coordinate (I01) at ($(M)+(-2,0)$);
\coordinate (Ia1) at ($(M)+(30:2)$);
\coordinate (In1) at ($(M)+(135:2)$);

\draw [thick, black!30!white] (I01) -- ($(I01)!1.25!(Ia1)$);
\draw [->, very thick, red] (0, 0) -- (Iki1);
\draw [->, very thick, red] (0, 0) -- (Ia1);
\draw [->, very thick, red] (0, 0) -- (In1);
\draw [->, very thick, red] (0, 0) -- (I01);
\draw [->, very thick, blue] (0, 0) -- (U1);
\draw [->, thick, red] (In1) -- (I01);

% Winkel
\draw [->, >=stealth] (I01)+(1,0) arc (0:15:1);
\coordinate (phi) at ($(0,0)!0.5!(In1)$);
\draw [->, >=stealth] (phi) let \p1 = (phi) in arc (45:92:({veclen(\x1,\y1)}););
%\draw [->, >=stealth] (0,0)+(0.5,0.5) arc (45:90:0.707);

% Beschriftung
\node at (M) [above] {$M$};
\node at (M) {$\times$};
\node at (U1) [left] {$\underline U_1$};
\draw (In1) +(-0.55, -0.15) node {$\underline I_{1N}$}
			+(-0.1, 0.25) node {$P_N$};
\draw (Ia1) +(-0.2, -0.35) node {$\underline I_{1A}$}
			+(0.2, 0.2) node {$P_A$};
\draw (Iki1)+(0, -0.25) node [left] {$\underline I_{1Ki}$}
			+(0, -0.25) node [right] {$P_\infty$};
\draw (I01)+(0, -0.25) node [left] {$\underline I_{10}$}
			+(0, -0.25) node [right] {$P_0$};
\draw (In1) +(0.05, -0.7) [font=\scriptsize] node {$\frac{\underline I_{2N}'}{1+\sigma_1}$}
			+(-0.1, 0.25) node {$P_N$};
\draw (I01)+(0.6,0.1) [right, font=\scriptsize] node {$\mu$};
\draw (phi)+(-0.15,0) [left, font=\scriptsize] node {$\varphi_{1N}$};
			
\end{tikzpicture}
%!tikz source end

\end{sectionbox}

\begin{sectionbox}
\subsubsection{Schlupfgerade}
\begin{cookbox}{Schlupfgerade $R_1 = 0\wedge R_\text{Fe}\neq 0$}
\item (Bei $R_\text{Fe} = 0$) Mittelpunkt $M$ auf -Im Achse
\item Schlupfgerade an beliebiger Stelle einzeichnen
\item gesuchtes $s$ aus Längenverhältnis zu bekanntem Schlupf bestimmen
\end{cookbox}
%!tikz editor 1.0
%!tikz source begin
\begin{tikzpicture}[>=angle 60, font=\small]

% Koordinatensystem
\draw [->, thick] (-.4, 0) -- (6, 0) node [above] {$-$Im};
\draw [->, thick] (0, -2.5) -- (0, 2.5) node [right] {Re};

% Kreis
\coordinate (M) at (3, 0);
\draw [thick] (M) circle (2);

% Vektoren
\coordinate (U1) at (0, 1.75);
\coordinate (Iki1) at ($(M)+(2,0)$);
\coordinate (I01) at ($(M)+(-2,0)$);
\coordinate (Ia1) at ($(M)+(30:2)$);
\coordinate (In1) at ($(M)+(135:2)$);
\coordinate (Pko) at ($(M)+(0,2)$);
\coordinate (Pku) at ($(M)+(0,-2)$);
\coordinate (sa) at ($(Iki1)!2!(Ia1)$);
\coordinate (s0) at ($(M)+(0.5, 0)$);

\draw [->, very thick, blue] (0, 0) -- (U1);
\draw [name path=pnline, thick, black!30!white] (Iki1) -- (In1);
\draw [name path=pkoline, thick, black!30!white] (Iki1) -- (Pko);
\draw [name path=pkuline, thick, black!30!white] (Iki1) -- (Pku);
\draw [name path=paline, thick, black!30!white] (Iki1) -- (sa);

\draw [name path=sline, thick, green!70!blue]  (s0) +(0, 2.5) -- +(0, -2.5);

\path [name intersections={of=sline and pkoline, by=sko}];
\path [name intersections={of=sline and pkuline, by=sku}];
\path [name intersections={of=sline and pnline, by=sn}];

\draw [thick, dashed] (Pko) -- (Pku);
\draw [thick] (M) ++(0, 0.25) arc (90:180:0.25);
\draw (M) ++(135:0.125) node [font=\tiny] {$\bullet$};

% Beschriftung
\node at (M) {$\times$};
\node at (U1) [left] {$\underline U_1$};
\draw (M) +(-0.2,0) node [below] {$M$};
\draw (In1) +(-0.1, 0.25) node {$P_N$}
			+(0,0) node {\textbullet};
\draw (Ia1) +(0.2, 0.2) node {$P_A$}
			+(0,0) node {\textbullet};
\draw (Iki1)+(0, -0.25) node [right] {$P_\infty$}
			+(0,0) node {\textbullet};
\draw (I01)+(0, -0.25) node [right] {$P_0$}
			+(0,0) node {\textbullet};
\draw (Pko) node [above] {$P_K$}
			+(0,0) node {\textbullet};
\draw (Pku) node [below] {$P_K$}
			+(0,0) node {\textbullet};
\draw (s0) node[red] {$\times$}
			+(0,-0.2) node [right] {$s_0 = 0$};
\draw (sko) node[red] {$\times$}
			+(0,0.1) node [right] {$s_K$};
\draw (sku) node[red] {$\times$}
			+(0,0) node [right] {$-s_K$};
\draw (sn) node[red] {$\times$}
			+(0,0.1) node [right] {$s_N$};
\draw (sa) node [right] {$s_A$};
\draw (s0) +(0,-2.3) node [right] {Schlupfgerade}; 		

\end{tikzpicture}
%!tikz source end

\end{sectionbox}

\begin{sectionbox}
\subsubsection{Maßstab}
\begin{symbolbox}
\begin{tabular}{p{2.8cm}cc}
Strommaßstab & $m_I$ & $\unitof{\si{\ampere\per\centi\meter}}$\\
Leistungsmaßstab & $m_P = m_1\cdot U_1\cdot m_I$ & $\unitof{\si{\watt\per\centi\meter}}$\\
Drehmomentmaßstab & $m_M = \frac{m_P}{2\pi\cdot n_\text{syn}}$ & $\unitof{\si{\newtonmeter\per\centi\meter}}$
\end{tabular}
\end{symbolbox}

\subsubsection{Ablesbare Werte}
\begin{emphbox}
  $R_1\neq 0\wedge R_\text{Fe}\neq 0$
\end{emphbox}
\begin{tabular}{p{4cm}l}
Aufgenommene elektrische Leistung & $P_1 = \overline{PD}\cdot m_P$\\
Eisenverluste Stator & $P_\text{Fe} = \overline{CD}\cdot m_P$\\
Kupferverluste Stator & $P_{\text{Cu}1} = \overline{BC}\cdot m_P$\\
Kupferverluste Rotor & $P_{\text{Cu}2} = \overline{AB}\cdot m_P$\\
Abgegebene mechanische Leistung & $P_m = \overline{PA}\cdot m_P$\\
Inneres Drehmoment & $M_i = \overline{PB}\cdot m_M$
\end{tabular}
\begin{symbolbox}
  Definition Punkt D: Orthogonale Projektion von $P$ auf Im-Achse\\
  \begin{tabularx}{\columnwidth}{lX}
  $R_1 = 0$ & $B = C$ und $M$ auf Höhe von $P_0$\\
  $R_\text{Fe} = 0$ & $C = D$ und $P_0$ auf -Im Achse
  \end{tabularx}
\end{symbolbox}

%!tikz editor 1.0
%!tikz source begin
\begin{tikzpicture}[>=angle 60, font=\small]

% Koordinatensystem
\draw [->, thick] (-.4, 0) -- (6, 0) node [above] {$-$Im};
\draw [->, thick] (0, -2) -- (0, 2.5) node [right] {Re};

% Kreis
\coordinate (M) at (3, 0.7);
\draw [thick] (M) circle (2);

% Vektoren
\coordinate (U1) at (0, 1.75);
\coordinate (P0) at ($(M)+(193:2)$);
\coordinate (Pinf) at ($(P0)+(13:4)$);
\coordinate (Pa) at ($(M)+(37:2)$);
\coordinate (Pn) at ($(M)+(130:2)$);
\coordinate (H) at ($(M)+(-13:2)$);
\coordinate (D) at ($(0,0)!(Pn)!(6,0)$);

\draw [->, very thick, blue] (0, 0) -- (U1);
\draw [name path= Cline, thick] (P0) -- (H);
\draw [name path= Bline, very thick, green!70!blue] (P0) -- (Pinf) node [black, pos=0.8, below, sloped] {Drehmomentlinie};
\draw [name path= Aline, very thick, green!70!blue] (P0) -- (Pa) node [black, pos=0.8, above, sloped] {Leistungslinie};
\draw [name path= dashedline, densely dashed] (Pn) -- (D);
\draw [thick] (M) -- +(105:2) node {$\bullet$} -- +(-75:2) node {$\bullet$};
\draw [black!50!white] (M) ++(105:0.2) arc (105:195:0.2);
\draw [black!50!white] (M) ++(150:0.1) node {.};

\path [name intersections={of=Aline and dashedline, by=A}];
\path [name intersections={of=Bline and dashedline, by=B}];
\path [name intersections={of=Cline and dashedline, by=C}];

% Beschriftung
\node at (M) {$\times$};
\node at (Pinf) {$\bullet$};
\node at (P0) {$\bullet$};
\node at (Pa) {$\bullet$};
\node at (Pn) {$\bullet$};
\node at (A) {$\bullet$};
\node at (B) {$\bullet$};
\node at (C) {$\bullet$};
\node at (D) {$\bullet$};
\node at (U1) [left] {$\underline U_1$};
\node at (Pn) [left] {$P_N$};
\node at (Pa) [right] {$P_A$};
\draw (M) ++(0.1,0) node [above] {$M$};
\draw (M)++(105:2) node [above] {$P_K$};
\draw (M)++(-75:2) node [below] {$P_K$};
\draw (Pinf) node [right] {$P_\infty$};
\draw (P0)  node [left] {$P_0$};
\draw (A) +(0, 0.12) node [left] {A};
\draw (B) +(0, 0.12) node [right] {B};
\draw (C) +(0, 0.08) node [right] {C};
\draw (D) +(0, -0.12) node [left] {D};
			
\end{tikzpicture}
%!tikz source end

\end{sectionbox}

\begin{sectionbox}
\subsection{Drehmoment}
\begin{emphbox}
$M_K\sim \left(\frac{U_1}{f_1}\right)^2\qquad M_N\sim\Phi_\delta\frac{U_1}{f_1}$
\end{emphbox}
\begin{align*}
M_i = M_R + M_L + J\frac{\partial\omega}{\partial t}
\end{align*}
\subsubsection{Drehmomentgleichung}
\begin{emphbox}
\[M_i = 3p(1-\sigma)\frac{{U_1}^2}{{\omega_1}^2 L_\sigma}\frac{s \cdot s_K}{\Delta\rho_1{s_K}^2 + 2\frac{\rho_1}{\sigma}(1-\sigma)s_K s+\Delta\rho_1 s^2}\]
\end{emphbox}
Kippmoment:\\
$M_K = M_i(s_K) = \frac{3}{2} p\cdot (1-\sigma)\frac{{U_1}^2}{{\omega_1}^2 L_\sigma}\left( \frac{1}{\Delta\rho_1+\frac{\rho_1}{\sigma}(1-\sigma)}\right)$\\
$(R_1 = 0): M_K = \frac{m_1U_1\frac{I_{1Ki} - I_{10}}{2}}{2\pi\cdot n_s}$\\
Kippschlupf: $s_K = \frac{\rho_2}{\sigma}\sqrt{\frac{1+{\rho_1}^2}{1+\left(\frac{\rho_1}{\sigma}\right)^2}}$

\begin{symbolbox}
\begin{tabularx}{\columnwidth}{CC}
$s_K > 0\quad$ Motor & $s_K < 0\quad$ Generator
\end{tabularx}
\end{symbolbox}
\end{sectionbox}

\begin{sectionbox}
\subsubsection{Klossche Gleichung (Annahme $R_1 = 0$)}
\begin{emphbox}
\[\frac{M_i}{M_K} = \frac{2\cdot s_K\cdot s}{{s_K}^2 + s^2}\]
\end{emphbox}
\[s_{1,2} = s_K \frac{M_K}{M_i} \pm \sqrt{\left(s_K \frac{M_K}{M_i}\right)^2 - {s_K}^2}\]
Nur echte Lösung wenn gilt: \quad $s < s_K$
\end{sectionbox}

\begin{sectionbox}
\subsection{Symmetrische Komponenten}
\begin{symbolbox}
\begin{tabularx}{\columnwidth}{>{\centering\arraybackslash}Xc>{\centering\arraybackslash}X}
$s_m + s_g = 2$ & $s_m = s = \frac{n_s - n}{n_s}$ & $s_g = \frac{n_s + n}{n_s}$
\end{tabularx}
\end{symbolbox}

\subsubsection{Spannungen Mit- und Gegensystem}
\begin{tabular}{lc}
Mitsystem & $\underline{U}_m = \frac{1}{3}\cdot (\underline{U}_u + \underline{a}\cdot\underline{U}_v + \underline{a}^2 \cdot \underline{U}_w)$\\
Gegensystem & $\underline{U}_m = \frac{1}{3}\cdot (\underline{U}_u + \underline{a}^2\cdot\underline{U}_v + \underline{a} \cdot \underline{U}_w)$\\
Nullsystem & $\underline{U}_m = \frac{1}{3}\cdot (\underline{U}_u + \cdot\underline{U}_v + \cdot \underline{U}_w)$
\end{tabular}\\
Nullsystem verschwindet bei Dreiecksschaltung oder Sternschaltung ohne herausgeführten Sternpunkt\\

\subsubsection{Drehmoment mit Kompensation (Kippschlupf ändert sich)}
\[M_\text{ges} = M_m - M_g\]
\begin{emphbox}
\[M = 3p\cdot (1-\sigma)\cdot\frac{{U_1}^2}{{\omega}^2 L_1} \cdot\frac{\rho_2 \cdot s}{(\rho_1\cdot\rho_2 -\sigma\cdot s)^2 +(\rho_2 +s\cdot\rho_1)^2}\]
\end{emphbox}
\end{sectionbox}
